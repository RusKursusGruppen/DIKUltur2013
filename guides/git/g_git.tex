\documentclass[11pt,a4paper,oneside,final,titlepage]{article}

%-- PACKAGE IMPORTS {{{1 --%

\usepackage[utf8]{inputenc}
\usepackage[T1]{fontenc}
\usepackage[english]{babel}
\usepackage[tmargin=1in]{geometry}
\usepackage{fancyhdr}
\usepackage{titlesec}
\usepackage{color}
\usepackage{graphicx}
\usepackage{lastpage}
\usepackage{amsmath,amssymb,amsfonts}
\usepackage{listings}
\usepackage{verbatim}
\usepackage{pifont}
\usepackage{caption}
\usepackage{mdwlist}
\usepackage{textcomp}
\usepackage{wasysym}
\usepackage{ulem}
\usepackage{enumerate}
\usepackage{fixltx2e}
\usepackage[hidelinks,pdftex]{hyperref}

%-- END PACKAGE IMPORTS }}}1 --%

%-- DEFINITIONS {{{1 --%

%---- Macros ----% {{{2
\newcommand{\todo}[1]{\color{red}\textbf{TODO}\hspace{0.1in}\textit{#1}\color{black}}
\newcommand{\remark}[1]{\small\color{yellow}\textbf{Remark}\hspace{0.1in}\textit{#1}\color{black}\normalsize}
\newtheorem{theorem}{Theorem}
\newcommand{\reference}[1]{$^{[#1]}$}

\renewcommand{\sectionmark}[1]{\markboth{#1}{}}
\renewcommand{\subsectionmark}[1]{\markright{#1}}
%---- End Macros }}}2 ----%

%---- Hyperlinks Setup {{{2 ----%

\hypersetup{
    % colorlinks     = true
    %,linkcolor      = cyan
    ,urlcolor       = cyan
    ,unicode        = true
    ,pdfauthor      = {Paw Saabye Pedersen}
    ,pdfsubject     = {Crash course in git}
    ,pdftitle       = {On Using GIT}
    ,pdfkeywords    = {git}{version control}{vcs}{revision control}
    ,pdfstartview   = {FitW}
}

%---- End Hyperlinks Setup }}}2 ----%

%---- Syntax Highlighting {{{2 ----%

%------ global syntax style {{{3 ------%
\lstset{
     backgroundcolor=\color{white}
    ,captionpos=b
    ,extendedchars=true
    ,tabsize=4
    ,aboveskip={1.5\baselineskip}
    ,columns=fixed
    ,upquote=true
    ,deletekeywords={}
    ,escapeinside={\%*}{*)}
    ,showspaces=false
    ,showtabs=false
    ,showstringspaces=false
    ,title=\lstname
    ,xleftmargin=\parindent
    ,basicstyle=\footnotesize
    ,numberstyle=\tiny\color[rgb]{0.5,0.5,0.5}
    ,identifierstyle=\ttfamily\color[rgb]{0.49,0.74,0.250}
    ,keywordstyle=\bfseries\ttfamily\color[rgb]{0,0,1}
    ,stringstyle=\ttfamily\color[rgb]{0.627,0.126,0.941}
    ,commentstyle=\textit{\color[rgb]{0,0.6,0}}
    ,frame=shadowbox
    ,frameround=ftft
    ,framerule=0.4pt
    ,framesep=3pt
    ,rulesep=2pt
    ,rulecolor=\color{black}
    ,rulesepcolor=\color{blue}
    ,breaklines=true
    ,resetmargins=false
    ,prebreak=\raisebox{0ex}[0ex][0ex]{\ensuremath{\hookleftarrow}}
    ,postbreak={}
    ,breakautoindent=true
    ,numbers=left
    ,stepnumber=5
    ,numbersep=5pt
}


% end global syntax style }}}3

%------ Language imports {{{3 ------%
\lstloadlanguages{Bash,sh}
%------ end language imports }}}3 ------%

\lstdefinestyle{git}{
    language=sh,
    stepnumber=1,
    morekeywords={mkdir,git,init,status,add,commit,--,reset,HEAD,checkout,diff,merge,clone}
}

%---- End Syntax Highlighting }}}2 ----%

%-- END DEFINITIONS }}}1 --%

%-- DOCUMENT PROPERTIES {{{1 --%
\pagestyle{fancy}

%---- Title {{{2 ----%
\title{
    \LARGE{DIKUltur}\\
    \huge{Guide to using git}
    \author{
        Paw Saabye Pedersen\\
        \texttt{paw@saabyepedersen.dk}
    }
    \date{\today}
}
%---- End Title }}}2 ----%

%---- Header and Footer {{{2 ----%
\rhead{\today \\ \leftmark}
\lhead{Paw Saabye Pedersen \\ git@dikultur.dk}
\chead{Guide to Using \textit{git} \\ \bfseries DIKUltur}
\cfoot{Page \thepage\ of \pageref{LastPage}}
%---- End Header and Footer }}}2 ----%

%-- END DOCUMENT PROPERTIES }}}1 --%

%-- DOCUMENT {{{1 --%
\begin{document}

\maketitle
\thispagestyle{empty}

\clearpage
\thispagestyle{empty}
\tableofcontents
\clearpage

\thispagestyle{fancyplain}
\setcounter{page}{1}

\section{Introduction}
Welcome to the wonderful world of git! A world in which you will most likely
--- and with good reason --- spend an ,,awful'', lot of time during your studies at
DIKU.

This guide is meant as a crash course in understanding this world, thus, enabling
\textit{you} to keep track of your files and projects, as well as collaborate
more efficiently with your fellow students.

\subsection{What is git}
Git is a \textbf{V}ersion \textbf{C}ontrol \textbf{S}ystem (VCS for short),
meaning that it --- at least conceptually --- records all your changes to the
files in a repository (repositories are discussed in section \ref{sec:repos}).\\
Why is this important, someone might ask (not you obviously), and the main
answer is that having all your changes stored enables you to revert back to a
previous version if you mess up your files, and that it enables you to more
efficiently discover mistakes, since you can see the differences between two
files of code.

\subsection{Repositories}
A repository is basically just a regular folder on your computer, which has a
VCS monitoring it. You will typically have an entire project in a repository,
rather than just single files scattered across multiple repositories. This will
--- most likely --- come natural as this is the normal structure of folders
containing code projects.
To turn a folder into a git repository, simply change directory to the given
project folder and initialise git, like so:

\begin{lstlisting}[name=lst:initialiseRepo,caption=Start monitoring a folder,
                   label=lst:initialiseRepo,style=git]
# Create a directory to demonstrate init.
 $  mkdir %*$\sim$*)/myGitProject
# Change directory to the one you just created
 $  cd %*$\sim$*)/myGitProject
# ... and initialise the git repo
 $  git init
\end{lstlisting}

You can always see the current repository status with either of the two following
commands:
\begin{lstlisting}[name=lst:status,caption=Check current repo's status,
                   label=lst:status,style=git]
# Check status along with descriptive text
 $  git status
# Or get only the status of the files
 $  git status -s
\end{lstlisting}

Files that you put in this new repository will not automagically be under
version control. You must explicitly tell git which files you wish for it to
monitor. This allows you to not share for instance compiled binary files
(which you will learn about in your programming courses, if you haven't already)
with other collaborators.\\
To tell git to monitor a file you ,,add it to the staging area'' (which will
hopefully become clear to you shortly). This can be done as follows:
\begin{lstlisting}[name=lst:addFiles,caption=Adding files to staging area,
                   label=lst:addFiles,style=git]
# Create a file to add
 $ echo "Some gibberish text" > foogittish.bar
# Add the file
 $ git add foogittish.bar
# Alternatively you can add all files in the folder
 $ git add %*\textbf{.}*)    # notice the dot!
\end{lstlisting}

\subsubsection{Staging area}
Even though we have now added our \lstinline[language=sh]{foogittish.bar} file
to the staging area, the changes we have made to it aren't stored in git.
The staging area is meant to be a place where you temporarily hold the files you
wish to publish. This allows you to publish (commit) several files under one.

Imagine, for instance, that you are working on your coursework for Object-Oriented
Programming and Design with your group, and that you have succesfully made a
class that inherits from another, yet to be written, class.\\
You wouldn't want to publish the inheriting class without its dependencies being
fulfilled, rather, you would want to publish the two --- along with any other
dependencies/deriative files --- together. The point, then, is that by grouping
together related files in what's called a \lstinline[style=git]{commit} you
can easily navigate your commit history, and know exactly which files had an
update when.

\subsubsection[label=sec:repos]{Commits}
\todo{Write this bit}
The aforementioned changes are stacked together in what's called a commits.

\subsubsection{Repositories}
\label{sec:repos}Repositories are basically just folders, which have a VCS
monitoring it.\\


\subsection{Version Control Systems (VCS)}
\remark{This section might be better off in its own file.} \\
\todo{This section needs to be written}\\
\remark{Short text to introduce that there will be a list of the most frequent
        vcs'. Perhaps also why Dropbox is \textbf{not} on the list!} \\

\subsubsection{Git}
\texttt{Pros and cons} \\
There are --- as always --- both pros of cons of git.

\subsubsection{SVN - Subversion}

\subsubsection{Mercurial}

\subsubsection{Others}
\remark{Explain that there are a lot of other systems out there, but that some
        of them are language specific (Perl's CTAN, Python's pip, Haskell's
        cabal} \\

\section{Collaborating}
When working on a project in a group you often find yourself needing to work on
the same codebase. Some people choose to make their own lives miserable and use
Dropbox, which, in my experience, has led to nothing good. You'll have multiple
working copies of a given file in folder, and you might even end up overwriting
several hours' work because someone saves an old copy of the same name. This is
where git \emph{really} comes into its right.

Normally when collaborating using git, you'll set up a remote repository on a
service like \mbox{\href{http://github.com}{github}},
\mbox{\href{http://bitbucket.com}{bitbucket}} or
\mbox{\href{http://assembla.com}{assembla}}, and you would do yourself a favor
creating a user there \textit{now}.

The author doesn't favour any of these services, but github is the most commonly
used service at DIKU. Creating a user there is fairly straight-forward and you
can use your mail address provided by KU to create a free user that let you
create up to 5 'private' repositories and an unlimited amount of 'public' ones.

\todo{Refrase following:} Once a new remote repository has been created it will have a link to it, which
utilises either the http(s) or the git protocol.

Once you've setup a remote repository you can make a local copy of its structure,
using the following command:
\begin{lstlisting}[label=lst:cloning,caption=How to clone a repository,
                   name=lst:cloning,style=git]
# Clone the remote repository into specified folder
 $  git clone "https://github.com/pspeder/dikultur-git.git" ./GittY
# You can also omit the folder name
 $  git clone "git@git://www.assembla.com/pspeder:dikultur-git.git"
# Which will name the repository the same as on the remote server
\end{lstlisting}
The above listing will setup all the necessary stuff needed to retrieve
(pull/fetch) and upload (push) your work to the server\footnote{We will discuss
how to push and pull a little later}. I shan't go into detail on how to set this
manually up here, but you may read about it in \href{http://scm-git.com/book/}
{git's own book}.

You can check that your local copy is pointing to the right location with the
command: \lstinline[style=git]{git remote -v}, which should produce two entries
both pointing to your remote server.

\section{Workflows}
\begin{lstlisting}[language=bash
                  ,label=lst:generalFlow
                  ,name=lst:generalFlow
                  ,caption=General workflow when working with git]
# Check which changes you made to your repository
$   git status
# If they're right add them to the staging area
# This can be done in one of two ways:
 # 1)
$   git add <the files you would like to add goes here>
 # 2)
$   git add .  # Add all changed files
$              # Note the dot.
Check that you added the correct files
$   git status
# If correct commit your staged files
$   git commit -m "<A message to go along with the commit>"
# If not remove the files from staging area with:
$   git checkout -- <files to be removed from staging area>
\end{lstlisting}
\end{document}
%-- END DOCUMENT }}}1 --%

